\documentclass[a4paper]{article}

% Support for Norwegian letters
\usepackage[latin1]{inputenc}
\usepackage[T1]{fontenc}
\usepackage{amsmath,amsfonts,amssymb,amsthm,booktabs,array,mathtools}
% consider package mhchem for typesetting chemical formulas

% Proper space and font for integral differential term
\newcommand{\dd}{\; \mathrm{d}} 
% Shorcut for ODEs with proper font
\newcommand{\diff}[2]{\frac{\mathrm{d} #1}{\mathrm{d} #2}}
% Shortcut for PDEs with proper font (shortcut: PDB)
\newcommand{\pdiff}[2]{\frac{\partial #1}{\partial #2}}
\newcommand{\pdiffn}[3]{\frac{\partial^{#3} #1}{\partial #2^{#3}}}

% Absolute value and norm commands.
% Read the mathtools.pdf to fix these!
\providecommand{\abs}[1]{\lvert#1\rvert} 
\providecommand{\norm}[1]{\lVert#1\rVert}

% Set the depth of section numbering
\setcounter{secnumdepth}{0}

\begin{document}

A poly(A) tail from an mRNA will be represented by so called polyA reads in
the output of an RNA-seq experiment. These polyA reads will primarily be on
the form $TTTTTT\ldots$, and they will map to the opposite strand of the
transcript from which it originated. The reason for this is that the original
mRNA has been turned into cDNA, and that sequencing happens from the 5' ends
of the ensuing cDNA pair. When the original mRNA is reverse-transcribed, the
poly(T) end is now 5' and is where sequencing begins from. Depending on the
read length and insert size of the experiment, there is also a probability of
obtaining polyA reads from the non-reversed fragment. For example, if the
insert size is 200 bt, and the poly(A) tail is 150 nt, then, with a read-length
of 75, the first 50 bases will map to the genome while the last 25 will be As.

In this example the poly(A) tail was 150 nt, while the full poly(A) tail is 250
nt. However, in the cDNA step, a dT primer is used, which binds at random in
the poly(A) tail. Thus the cDNA poly(A) tail will range in size from the dT
primer size (20) to the full length of 250.

All in all, whether we have polyA reads or polyT reads depends on 4 factors:
the dT cDNA step (distribution of poly(A) lengths), the insert size resulting
from the fragmentation, and the read length (# of sequenced nucleotides) from
both sides of the insert fragment.

A table showing the relationship between poly(A) and poly(T) reads.

\begin{tabular}{llr}
\toprule
\multicolumn{2}{c}{Item} \\
\cmidrule(r){1-2}
Animal & Description & Price (\$) \\
\midrule
Gnat  & per gram & 13.65 \\
      & each     &  0.01 \\
Gnu   & stuffed  & 92.50 \\
Emu   & stuffed  & 33.33 \\
Armadillo & frozen & 8.99 \\
\bottomrule
\end{tabular}

\bibliographystyle{plain}
\bibliography{/home/jorgsk/phdproject/bibtex/jorgsk}

\end{document}


