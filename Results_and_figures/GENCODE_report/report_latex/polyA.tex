\documentclass[a4paper]{article}

% Support for Norwegian letters
\usepackage[latin1]{inputenc}
\usepackage[T1]{fontenc}
\usepackage{amsmath,amsfonts,amssymb,amsthm,booktabs,array,mathtools}
% consider package mhchem for typesetting chemical formulas

% Proper space and font for integral differential term
\newcommand{\dd}{\; \mathrm{d}} 
% Shorcut for ODEs with proper font
\newcommand{\diff}[2]{\frac{\mathrm{d} #1}{\mathrm{d} #2}}
% Shortcut for PDEs with proper font (shortcut: PDB)
\newcommand{\pdiff}[2]{\frac{\partial #1}{\partial #2}}
\newcommand{\pdiffn}[3]{\frac{\partial^{#3} #1}{\partial #2^{#3}}}

% Absolute value and norm commands.
% Read the mathtools.pdf to fix these!
\providecommand{\abs}[1]{\lvert#1\rvert} 
\providecommand{\norm}[1]{\lVert#1\rVert}
\title{Two types of polyadenlyation of human RNA from RNA-seq of cellular
compartments}
\author{Jørgen Skancke}

% Set the depth of section numbering
\setcounter{secnumdepth}{0}

\begin{document} 
\maketitle

\subsection{Introduction}
Polyadenylation of RNA is pervasive in the animal kingdom. In bacteria,
polyadenylation marks RNA for degradation, while in eukaryotes mRNA are
polyadenylation as part of pre-mRNA processing, along with splicing and 5'
capping. 

Polyadenylation of eukaryotic messenger RNA (mRNA) happens co-transcriptionally
and is linked to transcription termination \cite{}. Both the site of
polyadenylation on pre-mRNA and the poly(A) tail itself affect the regulation
of transport and degradation \cite{}, \cite{}. In particular, in the cytoplasm
the poly(A) tail protects against degradation. A much commented difference
between eukaryotes and bacteria is that while polyadenylation protects against
degradation in eukaryotes, it facilitates degradation in bacteria. However,
recently reports have come of polyadenylation events in eukaryotes that are
linked to the degradation of RNA. Initially these reports were from yeast,
\cite{}, but recently reports have come of degrdation-related polyadenylation
in humans too \cite{}. In yeast the non-canonical polyadenylation has been
liked to the degradation of several types of transcripts, while in human they
have so far only been reported for ribosomal RNA (rRNA).

We used RNA-seq data from the ENCODE project to do a genome-wide profiling of
polyadenylation sites in the chromatin, nucleoplasmic, nuclear, and cytoplasmic
compartments.

Here we show evidence for non-canonical degradation-linked polyadenylation of
human RNA. Specifically, we show evidence of polyadenylation of introns in
the nucleus. Intronic polyadenylation would not be linked to transcript
termination, but rather to the degradation of these introns.

\subsection{Methods}
We mapped the RNA-seq data to the human genome and screened the unmapped reads
for leading T or trailing As, which were subsequently trimmed. The reads were
then remapped to the genome. The read was accepted as a read that stemmed from
a polyadenylation event if there was no poly(A/T) stretch on the region of the
genome which corresponds to the trimmed part of the read. The reads were also
screend against split-mapped reads to make sure that the poly(A) reads were not
just exon-exon junction reads over an A/T rich region.

The poly(A) reads were clustered together to form poly(A) clusters in the same
manner as \cite{}<++>. After clustering, we searched the downstream sequence of
each cluster for the polyadenylation signal (PAS). We also looked in a window
of 30 nucleotides around the polyadenylation site for an annotated
poly(A) site. The set of annotated poly(A) sites was obtained by merging the
poly(A) site annotation of GENCODE with the pA db\cite{} to obtain 43000
annotated polyadenylation sites.

Genomic regions were extracted from the GENCODE annotation version 7
\cite{}. Regions were extracted in a non-overlapping fashion with the following
precedence: exons > introns, 3UTR > 5UTR > CDS.

To map the reads to the genome we used the gem-mapper.

\subsection{Results}
\subsubsection{Datasets}
We investigated the HeLa, K563, and GM12872 cell lines from the ENCODE project
\cite{}. HeLa and GM12878 have RNA-seq data for the nuclear and cytosolic
compartments, both for poly(A)+ and poly(A)- RNA. For K562 there is also
chromatin and nucleoplasm data available, although for these compartments the
RNA has not been split into poly(A)+ and poly(A)- fractions.

\subsubsection{Canonical polyadenylation}
To confirm that we can pick up the transcription-termination related
polyadenylation (henceforth called canonical polyadenylation), we intersected
the polyadenylation sites we find with different genomic regions. The result
can be seen in figure 1. As can be seen, for the cytoplasmic compartments,
there is a strong enrichment of poly(A) clusters in the 3UTRs, a XXX fold
increase compared to the 5UTR and CDS combined. Of those poly(A) clusters that
land in the 3UTR, XXX\% land at annoated poly(A) sites and XXX\% have a PAS
signal associated with them. This confirms that the method of obtaining poly(A)
reads manages to cover annoated poly(A) sites. Since the RNA-seq method is not
optimized for 3' ends, we do not cover all poly(A) sites. The higher the RPKM
, the more frequent are poly(A) reads found for a given gene (data not shown).
Regardless, a good portion of poly(A) sites, both novel and annotated, are
found with these data. See Figure 2 for an intersection between the total
number of poly(A) sites found in the cytoplasmic regions and the annotated
poly(A) sites. See Table 1 for summary-statistics for the poly(A) clusters from
the different compartments.

\bibliographystyle{plain}
\bibliography{/home/jorgsk/phdproject/bibtex/jorgsk}

\end{document}


