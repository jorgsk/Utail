\documentclass[a4paper]{article}

% Support for Norwegian letters
\usepackage[utf8x]{inputenc}
\usepackage[T1]{fontenc}
\usepackage{amsmath,amsfonts,amssymb,amsthm,booktabs,array,mathtools}
% consider package mhchem for typesetting chemical formulas

% Proper space and font for integral differential term
\newcommand{\dd}{\; \mathrm{d}} 
% Shorcut for ODEs with proper font
\newcommand{\diff}[2]{\frac{\mathrm{d} #1}{\mathrm{d} #2}}
% Shortcut for PDEs with proper font (shortcut: PDB)
\newcommand{\pdiff}[2]{\frac{\partial #1}{\partial #2}}
\newcommand{\pdiffn}[3]{\frac{\partial^{#3} #1}{\partial #2^{#3}}}

% Absolute value and norm commands.
% Read the mathtools.pdf to fix these!
\providecommand{\abs}[1]{\lvert#1\rvert} 
\providecommand{\norm}[1]{\lVert#1\rVert}

\title{Polyadenlyation in spliced-out introns seen with RNA-seq of the
nucleoplasm of the K562 cell line}

% Set the depth of section numbering
\setcounter{secnumdepth}{0}

\begin{document} 

\subsection{Introduction} Polyadenylation of RNA is pervasive in the animal
kingdom. In eukaryotes messenger RNA (mRNA) is polyadenylated at the 3' end as
part of pre mRNA processing, and this poly(A) tail facilitates the transport
and stability of the mRNA \cite{colgan_mechanism_1997}. In bacteria,
polyadenylation is known for the opposite effect; the addition of a poly(A)
tail usually marks an RNA for degradation
\cite{mohanty_bacterial/archaeal/organellar_2011}. Recently, however,
polyadenylation as a signal for degradation was also identified in the yeast
nucleus \cite{lacava_rna_2005} \cite{wyers_cryptic_2005}. Here polyadenylation
has two roles: signalling for degradation in the nucleus and for protection
from degrdation in the cytoplasm Even more recently, polyadenylation that does
not come from pre mRNA processing has been found in human too
\cite{lutz_alternative_2011} and has also been linked to transcript degradation
both in the nucleus and cytoplasm \cite{slomovic_addition_2010}.

In several studies RNA-seq has been used to identify and study novel
polyadenylation sites in human, yeast, and worm \cite{fu_differential_2011}
\cite{ozsolak_comprehensive_2010} \cite{mangone_landscape_2010}. Since standard
RNA-seq protocols often give poor coverge of 3' ends, specialiced proceedures
were developed \cite{ozsolak_comprehensive_2010} \cite{fu_differential_2011}
\cite{fox-walsh_multiplex_????} to obtain a higher coverage of polyadenylation
reads. These studies have used whole cell extracts as substrates for the
sequencing. Since there is comparably less RNA in the nucleus than in the
cytoplasm, using whole cell extracts gives a comparably lower resolution to the
polyadenylation happening in the nucleus. Further, the cytoplasm and the
nucleus harbor different types of RNA. Connected to chromatin is nascent RNA,
but unprocessed and mid-processing (splicing and polyadenylation); in the
nucleoplasm are both successfully produces mRNA ncRNA, snoRNA etc, as well as
mis-processed RNA that is being degraded, and perhaps the largest RNA group in
the nucleoplasm are the  introns that are being degraded.

Here we investigate polyadenylation in human cell lines from the ENCODE project
in the the nucleic and cytoplasmic compartments separately. We find extensive
evidence of polyadenylation in 3UTRs both in the cytoplasm and the nucleus, and
we also find non-3UTR linked polyadenylation among other places in introns. We
speculate that the non-3UTR-linked polyadenylation signals result from
degradation-in-process.

\subsection{Methods}
We mapped the RNA-seq data to the human genome, screened and trimmed the
unmapped reads for leading T or trailing As, and remapped the trimmed reads to
the genome. A trimmed read was accepted as a read that stemmed from a polyadenylation
event if there was no poly(A/T) stretch on the region of the genome which
corresponds to the trimmed part of the read.

The reads were also screened against split-mapped reads to make sure that the
poly(A) reads were not just exon-exon junction reads over an A/T rich region.

The cleavage sites were clustered together to form poly(A) clusters (PAC) as in
\cite{tian_large-scale_2005}. We consider those PAC that have two or more
cleavage sites supporting them or fall at annotated cleavage sites. After
clustering, we searched the downstream sequence of each cluster for the
polyadenylation signal (PAS). We also looked in a window of 30 nucleotides
around the polyadenylation site for an annotated poly(A) site. The set of
annotated poly(A) sites was obtained by merging the poly(A) site annotation of
GENCODE with the polyAdb \cite{lee_polya_db_2007} to obtain 43000 annotated
polyadenylation sites.

The 5UTR, CDS, and 3UTR regions (and their introns) were extracted from the
GENCODE annotation version 7. Regions were extracted in a non-overlapping
fashion with the following precedence: exons > introns, 3UTR > 5UTR > CDS.

To map the reads to the genome we used the gem-mapper.
(http://gemlibrary.sourceforge.net/)

\subsection{Results}

\subsubsection{Identifying known polyadenylation cytoplasmic sites in the 3' UTR}
To verify that our method is able to find known polyadenylation sites, we
investigated the PAC that land in annotated 3UTR regions. Combining the
cleavage sites for  HeLA, K562, and GM12878, we obtained XXX clusters, of which
XXX land at annotated poly(A) sites, and XXY have PAS. Thus in addition to
identifying XXX annotated sites, we also identify XYX potentially novel sites,
YYX \% of which have a PAS. In figure 1a can be seen the cumulative number of
poly(A) sites obtained, as well as the relationship between the RPKM of
annotated 3UTRs with the number of poly(A) sites identified. As can be seen,
due to not using a 3'-optimized RNA-Seq protocol, the highly expressed
transcripts are most likely to have their poly(A) sites identified. However,
the ones that are found are of high specificity, since most land at annotated
poly(A) sites, and of those that do not, a high percentage have the PAS signal
downstream the putative cleavage site.

\subsubsection{Nucleoplasm and cytoplasm vs. whole cell}
Figure XXX shows the number of poly(A) sites in different genomic regions for
nucleoplasm, cytoplasm, and whole cell for K562. As can be seen, the whole cell
extract is most similar to the cytosolic extract. Most prominently, one sees
that the introns are poorly represented in the whole cell extract.

\subsubsection{Difference in poly(A) tail nucleotide composition and length
between introns and 3UTRs}

There is an enrichment of poly(A) reads in introns. In contrast to the poly(A)
reads in the 3UTR, few of these reads have the PAS signal (XX \%), which
supports the notion that these poly(A) signals are of a different nature than
the ones found at the 3' of mRNAs. By contrast, XYX \% of the the PAC in the
3UTR with only 1 poly(A) read have downstream PAS.

Further, the tail-lengths of the poly(A) reads is different. Also comment upon
the nr and distribution of A-runs and of T-runs. Seems the average length is
shorter in the 3UTRs!

TODO: compare the poly(A) reads without PAS in introns and 5'UTR exons. Are
there more or less in introns normalized for sequence length?

\subsection{Discussion}
It has recently become clear that human RNA is adenylated, likely in a transient,
degradation-linked manner CIT. If this adenylation has the same purpose as in
yeast and bacteria, adenylated tails may functions as docking stations for the
exosome, something which might be especially important if the RNA is structured
at the 3' end, preventing access to exonucleases CIT.

Why should we believe these results? Few because the process is transient.
The polyadenylated ones are quickly degraded, not stable like the other polyAz.
What do the polyadenylation events mean?

\subsubsection{Using standard RNAseq protocol for polyA}

Why do people want to find poly(A) for their dataset?
Instead of running a separate technique, they can apply their standard
protocol. 75 vs 150 etc. compare with that 36 read length one, the oldest. they
got very few. we get much more. how many with 150?


\bibliographystyle{plain}
\bibliography{/home/jorgsk/work/bibtex/biblib}
%\bibliography{/home/jorgsk/phdproject/bibtex/jorgsk}

\end{document}


